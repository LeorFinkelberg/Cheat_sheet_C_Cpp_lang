\documentclass[%
	11pt,
	a4paper,
	utf8,
	%twocolumn
		]{article}	

\usepackage{style_packages/podvoyskiy_article_extended}


\begin{document}
\title{Приемы программирования на языках C и C++}

\author{}

\date{}
\maketitle

\thispagestyle{fancy}

\tableofcontents

\section{Ресуры по языкам C и C++}

\url{https://learnc.info/c/}

\section{Указатели на функции}

\emph{Указатель} -- это переменная, содержащая \underline{\itshape адрес} другой переменной  Если одна переменная содержит {адрес} другой переменной, то говорят, что она \emph{указывает} на ту переменную\cite[\strbook{95}]{koltzov-c-lang:2019}.

Синтаксис объявления указателя
\begin{lstlisting}[
style = c_cpp,
numbers = none
]
тип *имя_указателя;
\end{lstlisting}

\texttt{тип} -- это тип переменной, на которую будет ссылаться указатель.

Указатель, не ссылающийся на конкретную ячейку памяти, должен быть равен \emph{нулю}. Использование нулевого указателя -- это всего лишь общепринятое соглашение
\begin{lstlisting}[
style = c_cpp,
numbers = none
]
// объявляем указатель на целочисленную переменную и инициализируем его с помощью NULL
int *p = NULL; 
\end{lstlisting}

\remark{
Указатель можно сравнивать с нулем или с \texttt{NULL}, но нельзя \texttt{NULL} сравнивать с переменной целого типа или типа с плавающей точкой \cite[\strbook{103}]{koltzov-c-lang:2019}
}

Несмотря на то, что функция \underline{не является переменной}, она располагается в памяти, и, следовательно, ее \underline{адрес} можно \emph{присваивать указателю}. Этот адрес считается точкой входа в функцию. Именно он используется при вызове. Поскольку \emph{указатель может ссылаться на функцию}, ее можно вызывать с помощью этого указателя. Это позволяет также передавать функцию другим функциям в качестве аргументов \cite{koltzov-c-lang:2019}.

\emph{Адрес функции} задается ее \emph{именем}, указанным без скобок и агрументов.

Пример
\begin{lstlisting}[
style = c_cpp,
numbers = none
]
// Подключение заголовочных файлов с помощью инструкции препроцессора #include
#include <stdio.h>
#include <string.h>
#include <stdlib.h>

// main - это точка входа приложения
int main() {
	char s1[10], s2[10]; // Объявление строк, как массива символов
	int (*p) (const char *, const char *); // Объявление указателя на функцию
	p = strcmp;  // Инициализация указателя; указателю присваивается адрес функции strcmp
	
	printf("Enter first string: ");
	scanf("%s", &s1);
	
	printf("Enter second string: ");
	scanf("%s", &s2);
	
	void test(
		char *x, // Указатель на символьную переменную; ожидает получить адрес
		char *y, // Указатель на символьную переменную; ожидает получить адрес
		int (*cmp) (const char *, const char *) // Указатель на функцию!
	) {
		// Сравнивает две строки
		printf("Comparation ...\n");
		if (!((*cmp)(x, y))) { // <===NB вызов функции strcmp как (*cmp)(x, y)
			printf("=> Equal!");
		} else {
			printf("=> Not equal.");
		}
	}
	
	// s1 и s2 это указатели на первый символ массива
	test(s1, s2, p);
	
	return 0;
}
\end{lstlisting}

Важный момент: \emph{имя массива} является \emph{указателем на его первый элемент} \cite[\strbook{78}]{koltzov-c-lang:2019}. То есть
\begin{lstlisting}[
style = c_cpp,
numbers = none
]
double *p; // Объявление указателя на вещественную переменную
double total[50]; // Объявление на массив вещественных чисел двойной точности
p = total; // Указателю p присваивается адрес первого элемента массива total
\end{lstlisting}

Поэтому в функцию \texttt{test} передаются не явные адреса \texttt{\&s1} и \texttt{\&s2}, а просто имена переменных \texttt{s1} и \texttt{s2}. Ведь имена переменных \texttt{s1} и \texttt{s2} связаны со строками (по сути массивами символов), а значит имена указывают на свои первые символы и таким образом в функцию на самом деле передаются \emph{адреса первых символов этих строк}. 

\section{Коротко о стеке и куче}

Стек -- это область памяти, которую вы, как программист, не контролируете никоим образом. В нее записываются переменные и информация, которые создаются в результате вызова любых функций. Когда функция заканчивает работу, то вся информация о ее вызове и ее переменные удаляются из стека автоматически.

Куча -- это область памяти, которую контролируют непосредственно программисты.

\section{Динамическое выделение памяти}

Благодаря \emph{динамическому выделению памяти} (dynamic allocation) программа может получать необходимую ей память в ходе выполнения, а не на этапе компиляции.

В языке C есть две функции динамического выделения памяти -- \texttt{malloc()} и \texttt{calloc()}. И одна функцию освобождения памяти -- \texttt{free()}.

Память, выделяемая функциями динамического распределения, находится в куче (heap), которая представляет собой область свободной памяти, расположенную между кодом программы, сегментом данных и стеком.

Прототип функции \texttt{malloc()}
\begin{lstlisting}[
style = c_cpp,
numbers = none
]
void *malloc(size_t количество_байтов)
\end{lstlisting}
Функция \texttt{malloc()} возвращает указатель типа \texttt{void *}. Это означает, что его можно присваивать указателю любого типа. В случае успеха функция \texttt{malloc()} возвращает \emph{указатель на первый байт памяти} (или другими словами, адрес зарезервированного участка памяти на куче), в противном случае (т.е. если размера кучи не достаточно для успешного выделения памяти) -- нулевой указатель (\texttt{NULL}).

Также полезна функция \texttt{calloc()}, позволяющая выделять память под данные конкретного типа данных
\begin{lstlisting}[
style = c_cpp,
numbers = none
]
void *calloc(size_t num, size_t size)
\end{lstlisting}

Размер выделенной памяти будет равен величине \verb|num * size|, где \verb|size| задается в байтах.

Следующий пример выделяет 2000 байт непрерывной памяти
\begin{lstlisting}[
style = c_cpp,
numbers = none
]
char *s; // объявление указателя на символьную переменную
s = malloc(2000);
\end{lstlisting}

После этого указатель \verb|s| будет ссылаться на первый из 2000 байт выделенной памяти.

Пример. Пусть требуется вычислить максимум в массиве, размер которого мы заранее не знаем
\begin{lstlisting}[
style = c_cpp,
numbers = none
]
#include <stdio.h>
#include <string.h>
#include <stdlib.h>

int main() {
	int i, num, size;
	float *data = NULL; // объявляем указатель на вещественную переменную
	
	printf("Enter number of elems: ");
	scanf("%d", &num);
	
	size = num * sizeof(float);
	
	// выделяем память
	if ((data = (float *) malloc(size)) != NULL) {
		printf("Allocated %d bytes of memory\n", size);
		// теперь с указателем data можно работать как с обычным массивом
		for (i = 0; i < num; ++i) {
			printf("Enter elem data[%d]=", i);
			scanf("%f", data + i); // `data + i` адресная арифметика
			/*
			Здесь можно было бы использовать нотацию []
			для обращения к элементам массива data, т.е.
			```
			...
			scanf("%f", &value);
			data[i] = value;
			```
			но тогда бы пришлось объявлять промежуточную переменную value;
			адресная арифметика позволяет передавать значения
			элементам массива без создания лишних переменных
			*/
		}
	} else {
		printf("Oops ...");
		exit(1);
	}
	
	for (i = 0; i < num; ++i) {
		if (*data < *(data + i)) { // значение первого элемента сравнивается с data[i]
			*data = *(data + i); // присваиваем новое значение первому элементу массива
		}
	}
	
	printf("Max(data)=%.2f", *data);
}
\end{lstlisting}

NB: оператор \verb|*| справа от символа присваивания это оператор разименования указателя (просто по адресу переменной получаем значение). А оператор \verb|*| слева от символа присваивания означает, что будет изменено значение соответствующей переменной.



% Источники в "Газовой промышленности" нумеруются по мере упоминания 
\begin{thebibliography}{99}\addcontentsline{toc}{section}{Список литературы}
	\bibitem{koltzov-c-lang:2019}{ \emph{Кольцов Д.М.} Си на примерах. Практика, практика и только практика. -- СПб.: Наука и Техника, 2019. -- 288 с.}
\end{thebibliography}

%\listoffigures\addcontentsline{toc}{section}{Список иллюстраций}

\lstlistoflistings\addcontentsline{toc}{section}{Список листингов}

\end{document}
