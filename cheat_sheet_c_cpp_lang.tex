\documentclass[%
	11pt,
	a4paper,
	utf8,
	%twocolumn
		]{article}	

\usepackage{style_packages/podvoyskiy_article_extended}


\begin{document}
\title{Приемы программирования на языках C и C++}

\author{}

\date{}
\maketitle

\thispagestyle{fancy}

\tableofcontents

\section{Указатели на функции}

\emph{Указатель} -- это переменная, содержащая адрес другой переменной  Если одна переменная содержит адрес другой переменной, то говорят, что она указывает на ту переменную\cite[\strbook{95}]{koltzov-c-lang:2019}.

Синтаксис объявления указателя
\begin{lstlisting}[
style = c_cpp,
numbers = none
]
int (*p) (const char *, const char *);  // указатель на функцию
// cout << asdfas;
тип *имя_указателя;
\end{lstlisting}

Несмотря на то, что функция не является переменной, она располагается в памяти, и, следовательно, ее \underline{адрес} можно \emph{присваивать указателю}. Этот адрес считается точкой входа в функцию. Именно он используется при вызове. Поскольку указатель может ссылаться на функцию, ее можно вызывать с помощью этого указателя. Это позволяет также передавать функцию другим функциям в качестве аргументов \cite{koltzov-c-lang:2019}.

\emph{Адрес функции} задается ее \emph{именем}, указанным без скобок и агрументов.



% Источники в "Газовой промышленности" нумеруются по мере упоминания 
\begin{thebibliography}{99}\addcontentsline{toc}{section}{Список литературы}
	\bibitem{koltzov-c-lang:2019}{ \emph{Кольцов Д.М.} Си на примерах. Практика, практика и только практика. -- СПб.: Наука и Техника, 2019. -- 288 с.}
\end{thebibliography}

%\listoffigures\addcontentsline{toc}{section}{Список иллюстраций}

\lstlistoflistings\addcontentsline{toc}{section}{Список листингов}

\end{document}
